\documentclass[11pt]{article}

\usepackage[utf8]{inputenc}
\usepackage[T1]{fontenc}
\usepackage{lmodern}
\usepackage{geometry}
\geometry{margin=25mm}
\usepackage{amsmath,amssymb}
\usepackage{booktabs}
\usepackage{hyperref}
\usepackage{microtype}

\hypersetup{
  colorlinks=true,
  linkcolor=blue,
  citecolor=blue,
  urlcolor=blue
}

% --- Notation -------------------------------------------------------------

\newcommand{\Min}{M_{\mathrm{in}}}
\newcommand{\SM}{S_{\mathrm{M}}}
\newcommand{\TL}{T_{\mathrm{L}}}
\newcommand{\VC}{V_{\mathrm{C}}}
\newcommand{\pC}{p_{\mathrm{C}}}
\newcommand{\UM}{U_{\mathrm{M}}}
\newcommand{\FC}{F_{\mathrm{C}}}
\newcommand{\XC}{X_{\mathrm{C}}}
\newcommand{\GC}{G_{\mathrm{C}}}

\newcommand{\K}[1]{\textsuperscript{K#1}}

% -------------------------------------------------------------------------

\title{Thermo-Credit Theory: A Credit-First Thermodynamic Mapping}
\author{ToppyMicroServices O\"U}
\date{Version 2.0 -- December 2025 (theory \& implementation update)}

\begin{document}
\maketitle

\begin{abstract}
We recast the Quantity Theory of Credit (QTC) in a thermodynamic-style state
space to separate scale, dispersion, and capacity in modern credit systems.
The key step is an entropy-like dispersion index that factors
money-in-circulation\K{10} and its allocation, combined with an explicit capacity
variable for bank balance sheets. This mapping is an analytic bookkeeping correspondence,
not a physical identity: it is designed for structured bookkeeping, stress
diagnostics, and early-warning indicators, not for importing physical laws
into economics. Within this framework we define an internal potential
$U(S_{\mathrm{M}}, V_{\mathrm{C}}, \ldots)$, derive a first-law-like decomposition of
credit creation, introduce a Helmholtz-style free energy $F_{\mathrm{C}}$ and an
exergy-like measure $X_{\mathrm{C}}$, and obtain a Maxwell-like integrability
condition that makes the mapping empirically falsifiable. The construction
is deliberately practice-first: all quantities are intended to be computable
from public data and to support decision-useful monitoring for investors,
risk managers, and policymakers.
\end{abstract}

\section{Introduction}

Classical money-first views such as the Quantity Theory of Money (QTM)
explain nominal dynamics in terms of a money stock, its velocity, and shocks
or policy actions \cite{Fisher1911,Friedman1956,Laidler1985}. More recent
credit-first views, notably the Quantity Theory of Credit (QTC), emphasize
that banks create deposits when they lend and that the use of credit (real
vs.\ financial, productive vs.\ speculative) matters for macro-financial
outcomes \cite{Werner2012,BoE2014}. In parallel, both information theory and
statistical mechanics have inspired analogies for income and wealth
distributions \cite{Theil1967,Foley1994,DragulescuYakovenko2000,Shannon1948,Jaynes1957}.


This note takes a minimal and operational step: we build a thermodynamic-style
state description that (i) separates scale and dispersion of
monetary/credit uses, (ii) introduces an explicit capacity variable for bank
balance sheets, and (iii) yields testable constraints and early-warning
gauges. The aim is not to claim that macro data obey physical laws, but to
provide a disciplined bookkeeping analogy useful for supervision, stress
testing\K{18}, and systematic monitoring.

\subsection*{Related Work}

Several strands of research motivate the thermodynamic-style mapping used in this note.  
First, high-frequency microstructure studies have explored temperature- and entropy-like summaries of order-book dynamics.  
For example, Li et al.\ (2024) define “market temperature’’ and “market entropy’’ from depth, imbalance, and turnover in limit-order books, showing that these quantities track short-term liquidity conditions and stress episodes.  
These approaches remain at the micro (tick-by-tick) level and do not involve credit allocation or regulatory constraints.

Second, Feng (2019) develops a thermodynamic representation of model risk using entropy flows and worst-case expectations.  
This work treats market uncertainty as an information-theoretic source of dissipation within pricing models.  
It is conceptually related through the shared use of temperature- and energy-like ideas, but it does not address credit capacity, bank balance sheets, or sectoral allocation.

The present note differs in scope: we apply a coarse-grained, macro-credit state description aimed at supervision and early-warning applications.  
Our variables $(S_M, V_C, T_L, p_C)$ are built from balance-sheet aggregates and dispersion measures, not from high-frequency order-book dynamics or pricing-model uncertainty.  
Loop area in the $(p_C, V_C)$ plane plays the role of dissipation at the macro-credit level, and the quality-factor construction introduced here has no direct analogue in existing microstructure or model-risk thermodynamic papers.

\section{Setup and Definitions}

We work over a chosen system (jurisdiction, sector set) and time aggregation
(e.g.\ monthly). Let $\Min$ denote money-in-circulation\K{10} over this domain,
and let $\{q_i\}$ be a stable, MECE (Mutually Exclusive, Collectively
Exhaustive)\K{15} partition of its uses (real activity, housing credit,
securities margin, etc.), with
\[
  \sum_i q_i = 1.
\]

% (Section break remains unchanged)
\subsection{Monetary dispersion entropy}

We define an entropy-like extensive index
\begin{equation}
  \SM = k \Min H(q), 
  \qquad
  H(q) \equiv - \sum_i q_i \log q_i,
  \label{eq:SM_def}
\end{equation}
where $k>0$ is a conventional scaling constant. This mirrors ideal mixing
entropy $\Delta S_{\mathrm{mix}} \propto N H(x)$ and follows the
information-theoretic form of Shannon \cite{Shannon1948}. $S_{\mathrm{M}}$
grows with both scale and dispersion; concentration (reallocation into fewer
uses) reduces $H(q)$ and can reduce $S_{\mathrm{M}}$ even at fixed $\Min$.

Formally, $H(q)$ is the Shannon--Boltzmann entropy of the MECE
bucket probabilities $\{q_i\}$ with natural logarithm (no Tsallis, R\'enyi,
or alternative families are used here).
In addition to the extensive $S_{\mathrm{M}}$, the implementation
tracks a scale-free version
$S_{\mathrm{M,hat}} = H(q)/\log K$ (with $K$ buckets) and per-bucket
contributions $S_{\mathrm{M,in}}^{(i)} = k M_{\mathrm{in}} \tilde q_i$.
For the baseline JP configuration these buckets correspond to
``productive'', ``housing'', ``consumption'', ``financial'', and
``government'' allocations (recorded as
${q_{\text{productive}},q_{\text{housing}},q_{\text{consumption}},
q_{\text{financial}},q_{\text{government}}}$); regional variants use
analogous coarse splits in their allocation tables, always normalised so
that $\sum_i q_i = 1$.

\paragraph{Energy and entropy conventions.}
The internal potential $U(S_{\mathrm{M}},V_{\mathrm{C}},\ldots)$ is a bookkeeping scalar in currency units, 
and the derived quantities $F_{\mathrm{C}}$, $X_{\mathrm{C}}$, and $G_{\mathrm{C}}$ are free-energy- and 
exergy-like gauges constructed from $(U,S_{\mathrm{M}},V_{\mathrm{C}})$ and fixed environmental parameters $(T_0,p_0)$. 
They are interpreted as indicators of usable credit capacity and policy room, not as literal physical energies; 
general caveats on the analogy are collected in Section~\ref{sec:limitations}.

\subsection{Capacity and conjugate variables}

We introduce:
\begin{itemize}
  \item $\VC$: an effective credit capacity or headroom\K{4} (e.g.\ CET1/RWA\K{3}
        slack, HQLA (High Quality Liquid Assets)\K{6}-based lending space,
        liquidity buffers\K{7});
  \item $\TL$: a liquidity-intensity index (a ``temperature-like'' proxy from
        spreads, turnover, depth);
  \item $\pC$: a ``credit pressure''\K{5} (shadow price of capacity).
\end{itemize}

We postulate an analytic potential
\begin{equation}
  U = U(\SM, \VC, \ldots),
  \label{eq:U_def}
\end{equation}
and define conjugate quantities
\begin{equation}
  \TL \equiv \left( \frac{\partial U}{\partial \SM} \right)_{\VC},
  \qquad
  \pC \equiv - \left( \frac{\partial U}{\partial \VC} \right)_{\SM}.
  \label{eq:conjugate}
\end{equation}
Here $U$ is not a physical internal energy; it is a bookkeeping potential
chosen so that a first-law-like decomposition holds. Classical QTM/QTC do
not name $U$; we introduce it as an analytic device.
At the level of the abstract mapping, \eqref{eq:conjugate} should be read
as a reference structure: in an idealized state description there exists
some smooth potential $U_{\mathrm{th}}(S_M,V_C)$ whose canonical fields
$(T_L^{\mathrm{th}},p_C^{\mathrm{th}})$ are given by these derivatives.
The empirical objects $U(t)$, $T_L(t)$, and $p_C(t)$ used in this note
are proxy series constructed from GDP-like aggregates, liquidity metrics,
and regulatory headroom; they are chosen to be interpretable and to track
the intended directions, but they are not literally defined by
finite-difference approximations to \eqref{eq:conjugate} in the code.

Empirically, $U(t)$ is instantiated as a scalar credit-capacity gauge constructed 
from GDP-like or credit aggregates. When a dedicated $U$ series is not available, 
the implementation fills it using a conservative fallback order over available GDP-like 
and credit-stock series (see Appendix, Internal-energy gauge). 
An optional detrended version $U^\star$ is used only for diagnostics and visualisation 
(to study fluctuations around a slowly varying baseline), 
while the core potentials $\FC$ and $\XC$ are computed from the undetrended $U$.

\subsection{Depth and turnover proxies}

In order to construct the liquidity temperature $\TL$ and related gauges
from observable quantities, we introduce two derived inputs:
a \emph{depth} proxy $D$ and a \emph{turnover} proxy $\Theta$.
They are not state variables on their own, but they are treated as
diagnostic inputs whose behaviour should be consistent with the thermodynamic
picture.

Conceptually, depth $D$ captures the scale of credit outstanding relative to
the system under study (e.g.\ private non-financial credit stocks) and
therefore measures how ``thick'' the market is.
In the absence of high-quality external depth series, we treat $D$ as a
rescaled version of real credit stocks $L_{\mathrm{real}}$,
so that, up to a calibration constant $d_0$,
\[
  D(t) \propto L_{\mathrm{real}}(t),
\]
with the constant chosen so that the median depth over the sample is near
the desired reference level.

Turnover $\Theta$ is intended to capture how quickly the system recycles
capacity: it is dimensionless and increases when credit capacity is deployed
and repaid more rapidly.
In the simplest case we use a ratio of the form
\[
  \Theta(t) \propto \frac{U(t)}{L_{\mathrm{real}}(t)},
\]
which compares a capacity-like quantity $U$ with the underlying stock
$L_{\mathrm{real}}$.
In practice this raw ratio is stabilised by replacing pathological values
(division by zero, missing data) with a reference level and clipping to a
reasonable band, so that extreme outliers do not dominate the temperature
construction.
These choices emphasise the interpretation of $\TL$ as a relative index
rather than an absolute physical temperature.

\paragraph{Design of the liquidity temperature.}
The liquidity temperature is designed to be high when spreads are narrow, markets are deep, and turnover is brisk, and to fall sharply when any one
of these ingredients becomes stressed.
To this end we combine standardized components
$(z_s,z_d,z_t)$---the z-scores of inverse spreads, inverse depth, and
turnover---into a composite
\[
  \hat T_{\mathrm{mult}}(t) = z_s(t)\,z_d(t)\,\bigl(1 + \tfrac{1}{2} z_t(t)\bigr),
\]
and then rescale to $T_{\mathrm{L}} \in [0,1]$ by min--max normalisation.
In what follows we use this multiplicative construction as the baseline and denote it simply by $\hat T$ when no confusion arises.
An additive alternative would be
\[
  \hat T_{\mathrm{add}}(t)
    = w_s z_s(t) + w_d z_d(t) + w_t z_t(t),
\]
with weights $w_s,w_d,w_t$; we prefer the multiplicative form at this
stage because it enforces a ``weakest link'' logic (a single severely
stressed component can pull the product down), while still allowing
turnover to modulate the combined index via the $(1+\frac{1}{2}z_t)$
factor.
The coefficient $\tfrac{1}{2}$ is a pragmatic choice that keeps turnover
influence comparable in scale to the spread/depth product without
dominating it.
Appendix-style robustness checks with $\hat T_{\mathrm{add}}$ and other
variants are natural next steps; empirically we treat the present
construction as a deliberately simple, economically interpretable starting
point rather than a uniquely correct functional form, and we do not use the additive alternative in baseline indicators.

\subsection{Units and scaling conventions}

All core quantities are defined to have consistent, but ultimately
conventional, units. The money stock $\Min$ and credit stocks such as
$L_{\mathrm{real}}$ are measured in currency units (e.g.\ trillions of
yen); the allocation shares $q_i$ are dimensionless, so the entropy
index $H(q)$ is measured in nats and $S_{\mathrm{M}} = k \Min H(q)$ has
units of ``currency $\times$ nats''. Since $H(q)$ is $\mathcal{O}(1)$ for
typical partitions, $S_{\mathrm{M}}$ scales essentially like $\Min$.

The capacity variable $\VC$ is expressed in the same monetary units as
the underlying balance-sheet aggregates (e.g.\ CET1-based headroom),
while the pressure $\pC = -(\partial U/\partial \VC)_{\SM}$ inherits
units of ``U per unit of volume''. In practice $U$ and $\VC$ are both
constructed from nominal stocks so that $\pC$ is dimensionless and
interpreted as an index of tightness; the liquidity temperature $\TL$ is
explicitly normalised to be dimensionless and lie in $[0,1]$.

When regulatory headroom components are available, a representative
example is the capital headroom
\[
  H_{\mathrm{cap}}(t)
    \approx V_R(t)\,\bigl(1 - \alpha_{\mathrm{cap}}\,p_R(t)\bigr),
\]
with $V_R$ a balance-sheet volume proxy and $p_R$ a regulatory pressure
proxy; analogous constructions apply for liquidity and funding headroom.
The effective capacity is then taken as
$\VC(t) = \min_j H_j(t)$ over such components.

The internal-energy gauge $U$ and the derived free energies $\FC$ and
exergy-like $\XC$ share the same nominal scale (currency units), up to
the choice of the scaling constants $k$, $T_0$, and $p_0$. These
constants are treated as knobs that set the overall magnitude of the
potentials rather than objects of primary interest: in applications we
focus on signs, relative levels, and changes (e.g.\ $\Delta \FC$,
$X_C^+$, $X_C^-$) rather than on any claim that the absolute values of
``credit energy'' carry direct economic meaning.

\section{Thermodynamic Correspondence (QTM vs.\ QTC)}

Table~\ref{tab:correspondence} summarizes the analogy. Thermodynamic
quantities are literal on the left column; QTM/QTC entries are analogy-level
constructs. Heat- and work-like pieces are defined for bookkeeping only.

\begin{table}[t]
  \centering
  \caption{Thermodynamic--QTM--QTC correspondence (analogy-level).}
  \label{tab:correspondence}
  \begin{tabular}{@{}llll@{}}
    \toprule
    Variable &
    Thermodynamics &
    QTM (money-first) &
    QTC (credit-first) \\
    \midrule
    Mixing entropy &
    $\Delta S_{\mathrm{mix}}$ &
    $\SM = k\Min H(q)$ &
    $\SM = k\Min H(q)$ \\
    Temperature &
    $T$ &
    $T_{\!L}$ (proxy) &
    $\TL$ (liquidity proxy) \\
    Internal energy &
    $U$ &
    $\UM(\SM)$ (auxiliary) &
    $U(\SM,\VC,\ldots)$ \\
    Volume &
    $V$ &
    -- &
    $\VC$ (capacity/headroom) \\
    Pressure &
    $p$ &
    -- &
    $\pC = -(\partial U / \partial \VC)_{\SM}$ \\
    Heat-like term &
    $\delta Q_{\mathrm{rev}} = T\,\mathrm{d}S$ &
    $Q_{\mathrm{M}} \sim T_0 \Delta \SM$ &
    $Q_{\mathrm{C}} \sim \bar{T}_{\!L}\,\Delta \SM$ \\
    Work-like term &
    $\delta W = p\,\mathrm{d}V$ &
    $W_{\mathrm{M}} \equiv W_{\mathrm{policy}}$ &
    $W_{\mathrm{C}} \equiv -\bar{p}_{\mathrm{C}} \Delta \VC
       + W_{\mathrm{policy}}$ \\
    First law &
    $\Delta U = T \Delta S + W$ &
    $\Delta \UM = T_0 \Delta \SM + W + \varepsilon$ &
    $\Delta U = \bar{T}_{\!L} \Delta \SM + W + \varepsilon$ \\
    Second-law tendency &
    $\Delta S \ge 0$ &
    $\Delta \SM \ge 0$ (mixing) &
    $\Delta \SM \ge 0$ (mixing) \\
    Helmholtz free energy &
    $F = U - TS$ &
    $F_{\mathrm{M}} \equiv \UM - T_0 \SM$ &
    $\FC \equiv U - T_0 \SM$ \\
    Exergy/availability &
    $X \approx U - T_0 S$ &
    $X \approx \UM - T_0 \SM$ &
    $\XC = \Delta U + p_0 \Delta \VC - T_0 \Delta \SM$ \\
    \bottomrule
  \end{tabular}
\end{table}

The construction is chosen so that: (i) $\SM$ is extensive and sensitive to
dispersion; (ii) $\VC$ carries explicit capacity/headroom information; (iii)
QTC---unlike classic QTM---naturally admits a pressure-like channel $\pC$.

\section{Bank Credit Creation and the Energy-Like Balance}

When a bank grants a new loan, it simultaneously creates a matching deposit.
On the joint bank--customer system, this is a pure accounting operation
(``loans create deposits''). Within our mapping, the associated change in the
potential $U$ is decomposed as
\begin{equation}
  \Delta U \approx \bar{T}_{\!L} \,\Delta \SM
               - \bar{p}_{\mathrm{C}} \,\Delta \VC
               + W_{\mathrm{policy}} + \varepsilon,
  \label{eq:first_law}
\end{equation}
where:
\begin{itemize}
  \item $\Delta \SM$: change in dispersion entropy from scale and allocation
        shifts (typically $\Delta \SM > 0$ in net expansion);
  \item $\Delta \VC$: change in effective capacity
        ($\Delta \VC < 0$ when headroom is used up; then
        $-\bar{p}_{\mathrm{C}} \Delta \VC > 0$);
  \item $W_{\mathrm{policy}}$: structured policy work\K{14} from regulation,
        guarantees, central bank operations, and other deliberate
        interventions;
  \item $\varepsilon$: residual term capturing model error and omitted
        channels.
\end{itemize}

Credit creation phases (new lending $>$ repayments) and contraction phases
(repayments $>$ new lending) then exhibit characteristic sign patterns for
$(\Delta \Min, \Delta \SM, \Delta \VC, -\bar{p}_{\mathrm{C}} \Delta \VC)$,
which can be tabulated and compared with data. The value of
\eqref{eq:first_law} is diagnostic: it forces us to attribute changes either
to dispersion, capacity use, or policy work\K{14}, rather than conflating
them. 

For intuition it is useful to summarize typical sign patterns.
Table~\ref{tab:sign_patterns} collects a minimal classification in terms
of net expansion/contraction and whether headroom is being used or
rebuilt; the qualitative readings in the caption are meant as guides
rather than axioms.
\begin{table}[t]
  \centering
  \caption{Illustrative sign patterns in expansion/contraction regimes.
  ``Expansion with headroom use'' corresponds to broad-based credit
  growth ($\Delta \Min>0$, $\Delta \SM>0$) that eats into capacity
  ($\Delta \VC<0$). ``Expansion with headroom rebuild'' captures cases
  where asset-side growth and recapitalisation outpace use
  ($\Delta \Min>0$, $\Delta \VC>0$). ``Contraction with headroom rebuild''
  refers to deleveraging or freezing of activity while buffers are
  restored ($\Delta \Min<0$, $\Delta \VC>0$). ``Contraction with headroom
  use'' flags stressed regimes in which activity shrinks yet capacity is
  still eroded ($\Delta \Min<0$, $\Delta \VC<0$; $\Delta \SM$ can be
  ambiguous).}
  \label{tab:sign_patterns}
  \begin{tabular}{@{}llll@{}}
    \toprule
    Regime & $\Delta \Min$ & $\Delta \SM$ & $\Delta \VC$ \\
    \midrule
    Expansion with headroom use
      & $+$ & $+$ & $<$\,0 \\
    Expansion with headroom rebuild
      & $+$ & $\gtrsim 0$ & $>$\,0 \\[2pt]
    Contraction with headroom rebuild
      & $<$\,0 & $\le 0$ & $>$\,0 \\
    Contraction with headroom use
      & $<$\,0 & ambiguous & $<$\,0 \\
    \bottomrule
  \end{tabular}
\end{table}
These patterns provide a qualitative checklist against which observed
$(\Delta \Min,\Delta \SM,\Delta \VC)$ can be compared, and can be refined
or extended in empirical work.

In addition to these pointwise diagnostics, we can study the \emph{loop
area} traced out by trajectories in the $(\pC,\VC)$ plane over policy or
credit cycles.
In the reversible limit the line integral
\[
  W_{\mathrm{loop}} = \oint_{\mathcal{C}} \pC\,\mathrm{d}\VC
\]
would vanish for closed cycles $\mathcal{C}$, reflecting path-independence
of the underlying potential. In practice, non-zero loop area measures
irreversibility and dissipative effects analogous to hysteresis in magnetic
systems or frictional cycles. The streaming estimator in the code updates
an exponentially weighted version of this integral via
\[
  W_t = \lambda W_{t-1}
        + \pC(t-1)\bigl(\VC(t)-\VC(t-1)\bigr),
\]
with forgetting factor $\lambda$ chosen so that old cycles gradually lose
influence.

Large or persistent values of $W_t$ therefore flag regimes where policy
paths in $(\pC,\VC)$ space exhibit pronounced hysteresis and stress.

\subsection{Dissipation and a credit quality factor $Q$}

In the data, irreversibility shows up as non-zero loop area in the
$(p_{\mathrm{C}},V_{\mathrm{C}})$ plane.
In physics, a similar idea is captured by the quality factor $Q$ of an
oscillator: it compares the energy stored in the system with the energy
lost in one cycle. Here we define an analogous quality factor for the
credit system.

Let $\mathrm{loop\_area}(t)$ be the dissipative part of the policy or
credit cycle in period $t$ and let $X_{\mathrm{C}}(t)$ be the
exergy-like ``usable credit capacity'' at time $t$.
We first define a simple loss ratio
\begin{equation}
  \phi(t) = \frac{\mathrm{loop\_area}(t)}{X_{\mathrm{C}}(t-1)},
\end{equation}
whenever $X_{\mathrm{C}}(t-1) > 0$.
When $X_{\mathrm{C}}(t-1) \le 0$, $\phi(t)$ and $Q_{\mathrm{C}}(t)$ are left undefined (treated as missing) in the baseline implementation.
This tells us what fraction of available exergy is lost through policy
hysteresis in that period.
We then define the credit quality factor
\begin{equation}
  Q_{\mathrm{C}}(t)
    = \frac{2\pi}{\phi(t)}
    = 2\pi \, \frac{X_{\mathrm{C}}(t-1)}{\mathrm{loop\_area}(t)}.
\end{equation}

A large $Q_{\mathrm{C}}$ means that the system loses only a small share of
its usable credit capacity in each cycle: policy loops move credit around
with limited long-run damage.
A small $Q_{\mathrm{C}}$ means that even modest policy loops destroy a
non-trivial part of $X_{\mathrm{C}}$, which is consistent with a fragile
or strongly path-dependent credit structure.

For mechanical oscillators, a standard first-order approximation is
\(
  \Delta E / E \approx -2\pi/Q.
\)
By analogy, we can write
\begin{equation}
  \frac{\Delta X_{\mathrm{C}}(t)}{X_{\mathrm{C}}(t-1)}
    \approx - \frac{2\pi}{Q_{\mathrm{C}}(t)}
    + \mathrm{external\_injection}(t) + \varepsilon_t,
\end{equation}
where the external-injection term collects net additions of usable
credit capacity (for example, through large-scale asset purchases or
recapitalisation episodes).
In this way $Q_{\mathrm{C}}$ acts as a simple, dimensionless summary of
how much hysteresis the credit system can tolerate before its usable
capacity is eroded.

\paragraph{Interpreting $W_{\mathrm{policy}}$.}
In the abstract decomposition \eqref{eq:first_law}, $W_{\mathrm{policy}}$
collects structured interventions (policy-rate shifts, large-scale asset
purchases, regulatory changes) that are not captured by the dispersion and
capacity-use terms.
In the reference implementation we do not attempt to estimate
$W_{\mathrm{policy}}(t)$ separately inside the indicator pipeline; instead
we compute the finite-difference terms
$\Delta U$, $\bar T_{\!L}\Delta \SM$, and $-\bar p_{\mathrm{C}}\Delta \VC$
and treat the residual $\Delta U - (\bar T_{\!L}\Delta \SM - \bar
p_{\mathrm{C}}\Delta \VC)$ (the ``first-law residual'') as a combined bucket
for policy work and model error.
For case studies or event analyses, users can overlay simple parametric
forms---for example, specifying
$W_{\mathrm{policy}}(t) = \gamma \mathbf{1}\{t \in [t_0,t_1]\}$ around a
major programme such as a large-scale asset purchase episode, or using
regression-style dummies---to attribute part of the residual to known
policy actions while leaving the remainder as $\varepsilon$.

\section{Free Energy, Exergy, and Early-Warning Gauges}

We seek a scalar gauge that (i) decreases as dispersion rises under a fixed
environment, and (ii) upper-bounds structured work extractable over a cycle.
A Helmholtz-style free energy provides this.

For a fixed reference environment $(T_0)$, define
\begin{equation}
  \FC \equiv U - T_0 \SM.
  \label{eq:FC_def}
\end{equation}
Then, under suitable regularity conditions,
\begin{equation}
  \mathrm{d} \FC = - \pC \,\mathrm{d} \VC + \delta W_{\mathrm{other}},
  \label{eq:dFC}
\end{equation}
so that $\FC$ plays the role of an available-potential measure. As
$\Delta \FC \to 0$ under chosen boundaries, policy headroom for structured
adjustments is effectively exhausted.

When an ambient pressure-like parameter $p_0$ is also relevant, an
exergy-like functional is
\begin{equation}
  \XC = (U - U_0) + p_0 ( \VC - V_{\mathrm{C},0} )
        - T_0 ( \SM - S_{\mathrm{M},0} ),
  \label{eq:XC_def}
\end{equation}
which reduces to $-\Delta \FC$ if $\VC$ is fixed and $p_0$ effects are
negligible. Because $\XC$ depends on boundary choices $(T_0, p_0)$, we treat
it as an optional, environment-dependent gauge.

In principle $\XC$ can take either sign, depending on the chosen
reference state $(U_0,V_{\mathrm{C},0},S_{\mathrm{M},0})$ and
environmental parameters $(T_0,p_0)$.
Negative values indicate that the current configuration has lower
``available credit energy'' than the reference; for early-warning
visualisation, however, it is often convenient to enforce a
non-negative floor (by clipping or by subtracting the minimum value over
the sample) so that zero marks the effective boundary of usable
headroom.

For regime diagnostics it is also useful to separate surplus and
shortage components relative to a fixed free-energy reference.
Let $\Delta F_C(t) = F_C(t) - F_C^{\mathrm{ref}}$ for some chosen
baseline $F_C^{\mathrm{ref}}$ (e.g.\ the first non-missing observation
or a user-specified date).
We then define
\[
  X_C^+(t) = \max\{0,\;\Delta F_C(t)\},
  \qquad
  X_C^-(t) = \max\{0,\;-\Delta F_C(t)\},
\]
so that $X_C^+$ tracks ``excess'' free energy above the reference and
$X_C^-$ tracks shortfall.
These series enter plots and summary tables as complementary early-warning
gauges: persistent build-up of $X_C^+$ suggests unusually large policy
room relative to the reference regime, whereas elevated $X_C^-$ points
to sustained pressure towards the boundary of feasible adjustment.

A Gibbs-style free energy
\begin{equation}
  \GC \equiv U + p_0 \VC - T_0 \SM
  \label{eq:GC_def}
\end{equation}
can be used when $(T_0, p_0)$ are the natural controls. In applications, we
typically monitor $\FC$ (fixed environment) and compare with $\XC$ or $\GC$
when capacity constraints are explicit (cf.\ exergy notions in
\cite{Wall1977}).


\paragraph{External coupling.}
In applications we also allow the credit system to be coupled to an
external ``pressure bath'' and ``temperature bath''.
The coupling is constructed in two stages.
First, we map a small set of market driver series into standardized
monthly processes.
For JP, the pressure drivers are conceptually:
\begin{itemize}
  \item the US high-yield option-adjusted spread (HY OAS), used in level
        form as a proxy for global credit stress;
  \item the US--JP 10Y yield spread,
        $\Delta y_{US,JP}(t) = y_{US}^{10\mathrm{Y}}(t) - y_{JP}^{10\mathrm{Y}}(t)$;
  \item USD/JPY log returns,
        $r_{\mathrm{FX}}(t) = \log \mathrm{USDJPY}(t)
        - \log \mathrm{USDJPY}(t-\Delta t)$.
\end{itemize}
The temperature driver is the VIX equity volatility index, again mapped
to monthly frequency.
Each raw series $X_k(t)$ is aggregated to month-start, transformed
according to its role (level, spread, log-return), and converted to a
z-score
\[
  Z_k(t) = \frac{X_k(t) - \mathbb{E}[X_k]}{\sigma[X_k]},
\]
using sample mean and standard deviation over the available history.
Second, we form composite indices by simple averaging over the available
standardized drivers:
\[
  E_p(t) = \frac{1}{K_p} \sum_{k \in \mathcal{P}} Z_k(t),
  \qquad
  E_T(t) = \frac{1}{K_T} \sum_{k \in \mathcal{T}} Z_k(t),
\]
where $\mathcal{P}$ indexes pressure drivers and $\mathcal{T}$ indexes
temperature drivers.
Missing values in individual $Z_k$ are ignored on each date so long as at
least one driver remains.
Equal weights reflect an agnostic prior about the relative importance of
each driver: with only a handful of series, more elaborate factor models
or time-varying loadings would be weakly identified and risk overfitting,
so the unweighted mean serves as a transparent, robust first pass.
In future work one could replace this averaging with a statistically
estimated factor structure or region-specific loadings once longer
histories and stronger identification are available.

Given $E_p$ and $E_T$ we work with effective fields
\[
  p_{\mathrm{C}}^{\mathrm{eff}}(t)
    = p_{\mathrm{C}}(t) + \alpha E_p(t),
  \qquad
  T_{\mathrm{L}}^{\mathrm{eff}}(t)
    = T_{\mathrm{L}}(t) + \delta E_T(t),
\]
so that external stress and risk-on/off conditions act as linear
perturbations to the internal equation of state.
When $\alpha$ or $\delta$ are set to zero the system is ``isolated'' in
this sense; non-zero couplings encode how exposed the credit state is to
external financial conditions.

\paragraph{Sectoral ``chemical potentials''.}
Given the MECE allocation shares $\{q_i\}$ and $M_{\mathrm{in}}$, we can
form sectoral analogues of chemical potentials.
Using the normalised shares $\tilde q_i$ from the entropy construction,
one defines
\[
  \mu_i
    = T_0 k M_{\mathrm{in}} \bigl( \log \tilde q_i + 1 \bigr),
\]
so that, up to constants, $\mu_i$ is the derivative of the entropy-like
term with respect to the bucket mass.
For interpretation we work with deviations from the cross-bucket mean,
\(
  \Delta\mu_i = \mu_i - \bar\mu
\),
where $\bar\mu$ is the average over $i$ at each date.
Large positive $\Delta\mu_i$ flag categories where incremental ``credit
mass'' is unusually valuable in entropy terms; large negative values flag
categories where allocations are already heavy relative to the rest of
the system.
In the current implementation these $\mu_i$ and $\Delta\mu_i$ are used as
diagnostic series and are not themselves fed back into the flow dynamics.

\section{Maxwell-Like Relation, Exterior Derivative, and Legendre Structure}

In the idealized reference model where $U$ is a well-defined state
potential, mixed partial derivatives commute. Using the definitions of
$\TL$ and $\pC$ in
\eqref{eq:conjugate}, we obtain a Maxwell-like condition:
\begin{equation}
  \left( \frac{\partial \TL}{\partial \VC} \right)_{\SM}
  = - \left( \frac{\partial \pC}{\partial \SM} \right)_{\VC}.
  \label{eq:Maxwell}
\end{equation}

Geometrically, the pair $(\TL,\pC)$ defines an exact differential
one-form on the $(\SM,\VC)$ state space:
\[
  \mathrm{d}U_{\text{rev}} = \TL\,\mathrm{d}\SM - \pC\,\mathrm{d}\VC,
\]
obtained by removing the explicitly non-conservative contributions
($W_{\mathrm{policy}}$ and $\varepsilon$) from \eqref{eq:first_law}.
Exactness of this one-form is equivalent to the vanishing of its exterior
derivative $d\omega = 0$ (no ``curl''),
which in coordinates reads
\[
  \frac{\partial}{\partial \VC}
  \left( \frac{\partial U}{\partial \SM} \right)_{\VC}
  =
  \frac{\partial}{\partial \SM}
  \left( \frac{\partial U}{\partial \VC} \right)_{\SM},
\]
and, after substituting \eqref{eq:conjugate}, yields
\eqref{eq:Maxwell}. In other words, the Maxwell-like equality is the
integrability condition for reconstructing a single potential $U$ from
its components $(\TL,\pC)$.

This also clarifies the Legendre structure.
Starting from $U(\SM,\VC)$ we can, in principle, define a potential
written in terms of the intensive variables,
\[
  \Phi(\TL,\pC)
    \equiv U(\SM,\VC) - \TL \SM + \pC \VC,
\]
where $(\SM,\VC)$ are now regarded as functions of $(\TL,\pC)$ via the
equation of state. Taking the differential and using
\eqref{eq:conjugate} one finds
\[
  \mathrm{d}\Phi = - \SM\,\mathrm{d}\TL + \VC\,\mathrm{d}\pC,
\]
so that
\(
  (\partial \Phi / \partial \TL)_{\pC} = -\SM
\)
and
\(
  (\partial \Phi / \partial \pC)_{\TL} = \VC
\).
This Legendre transform from $(\SM,\VC)$ to $(\TL,\pC)$ is globally
well-defined only when \eqref{eq:Maxwell} holds; the free-energy-type
potentials introduced in \S5 (such as $\FC$ and $\GC$) can then be
viewed as partially Legendre-transformed versions of $U$ evaluated at
fixed environmental parameters $(T_0,p_0)$.

Empirically, one can estimate $\TL(\SM,\VC)$ and $\pC(\SM,\VC)$ from proxy
series and test whether \eqref{eq:Maxwell} approximately holds.
Because $U$, $T_L$, and $p_C$ are constructed from partially independent
proxies rather than being derived from a single fitted potential, the
relation is not enforced by construction; instead it functions as an
integrability diagnostic. Persistent, large violations indicate that the
chosen variables are not behaving in a state-like way or that the proxy
mapping is misspecified. In this sense, the analogy is falsifiable rather
than purely rhetorical, and Section~\ref{sec:insights} summarizes a simple
set of operational diagnostics built from this condition.

\section{Insights and Practical Tests}\label{sec:insights}

The mapping yields several operational diagnostics:
\begin{enumerate}
  \item \textbf{Integrability test.} Estimate $\TL$ and $\pC$ as functions of
        $(\SM,\VC)$ and test \eqref{eq:Maxwell}. Failure suggests
        mis-specification of capacity or dispersion metrics.

  \item \textbf{Work vs.\ dispersion decomposition.} Use \eqref{eq:first_law}
        to decompose changes in $U$ into heat-like dispersion
        $(\bar{T}_{\!L}\Delta \SM)$, capacity use
        $(-\bar{p}_{\mathrm{C}} \Delta \VC)$, and policy work
        $(W_{\mathrm{policy}})$.

  \item \textbf{Free-energy and exergy ceilings.} Monitor $\FC$ or $\XC$ as
        early-warning gauges: sustained proximity to zero under stress
        scenarios flags limited room for non-disruptive adjustment.

  \item \textbf{Loop area and hysteresis.} Compute loop areas in $(\pC,\VC)$
        over policy cycles. Non-zero areas capture dissipative stress and
        irreversibility in credit allocation dynamics\K{17}.
\end{enumerate}

All of these rely on observable or reconstructible quantities from public
balance-sheet and market data. Implementation details (exact proxy choices,
normalization, robustness checks) are part of ongoing empirical work and
should be reported alongside results.

\section{Limitations and Scope}\label{sec:limitations}
This construction is intentionally modest. Key limitations include:
\begin{itemize}
  \item \textbf{Category dependence.} $\SM$ depends on the chosen partition
        of uses. Robustness across reasonable partitions must be checked.

  \item \textbf{Proxy noise.} $\TL$, $\VC$, $\pC$ are built from noisy
        proxies, and $U$ itself is instantiated from GDP- or
        credit-like aggregates. Measurement error or proxy misspecification
        can break \eqref{eq:Maxwell} and distort the decomposition
        \eqref{eq:first_law}; in practice these relations are treated as
        reference structures and integrability checks rather than hard
        identities.

  \item \textbf{Quasi-static approximation.} Fast crises and regime shifts
        violate smooth state-variable assumptions; the mapping is best seen
        as quasi-static or coarse-grained.

  \item \textbf{Non-physical.} We do not assume microscopic ``money
        particles'' or claim that macro-financial data obey physical first/
        second laws. The mapping is analogy-level and judged only by
        empirical usefulness.

  \item \textbf{Identification challenges.} Policy, expectations, and shocks
        act jointly. Causal claims require careful research design beyond
        the bookkeeping framework presented here.

  \item \textbf{Scaling conventions.} Constants such as $k$ are
        conventional. Report normalized metrics and sensitivity checks.
\end{itemize}

This version is primarily a theory-and-implementation note; a more systematic
empirical evaluation for the JP configuration will follow in a later revision.

\section{Notes on Reproducibility and AI Assistance}

This note is designed to accompany an openly available codebase and data
snapshots (e.g.\ monthly reports computed from public sources). The
reference implementation lives in the repository
\texttt{2025\_11\_Thermo\_Credit} and is intended to make all reported
figures and indicators reproducible.

At a high level:
\begin{itemize}
  \item a Python pipeline computes $\SM$, $\TL$, $\VC$, $\FC$, $\XC$, and
        the diagnostics from public time series using shared configuration
        files and scripts;
  \item minimal synthetic CSVs are provided so that tests and examples can
        run without external API keys, while real data are fetched from
        standard sources (e.g.\ FRED, World Bank) when credentials are
        present;
  \item continuous-integration workflows (GitHub Actions) pin Python
        dependencies and re-run the pipeline so that indicators and the
        report can be regenerated from a fixed environment.
\end{itemize}
Further implementation details (module structure, configuration roles, CI
layout, and synthetic data conventions) are documented in the repository
README and configuration files, and can be treated as an implementation
guide for practitioners wishing to reproduce or extend the system.

Drafting and editing of the present text used a GPT-based large language
model as a tool at low randomness. All equations, definitions, and claims
are curated and are the responsibility of the authors. Readers should rely
on the archived PDF and associated repository for the citable version and
reproducible code.

\bigskip

\noindent\textbf{Disclaimer.}
This material is part of an ongoing research program. It is provided ``as
is'' without any warranty of accuracy, completeness, or fitness for a
particular purpose. It does not constitute investment, legal, tax, or policy
advice, and does not create a client relationship or any regulated financial
service. Users are responsible for compliance with applicable laws in their
jurisdiction.

\appendix

\section*{Appendix: Implementation Notes (Modules \& CI)}

This appendix summarizes the main implementation modules for readers who
wish to reproduce or extend the reference pipeline.
It is not required for understanding the theoretical sections.

\begin{itemize}
  \item \textbf{Raw and feature tables.} Region-specific builder scripts
        \texttt{scripts/01\_build\_features.py} (JP),
        \texttt{scripts/04\_build\_features\_eu.py} (EU), and
        \texttt{scripts/05\_build\_features\_us.py} (US) select source
        series according to configuration files
        (\texttt{config.yml}, \texttt{config\_\{jp,eu,us\}.yml}) and
        \texttt{lib/series\_selector.py}. They write money, credit,
        regulatory-pressure, and allocation tables under \texttt{data/}
        (e.g.\ \texttt{money*.csv}, \texttt{credit*.csv},
        \texttt{reg\_pressure*.csv}, \texttt{allocation\_q*.csv}).
  \item \textbf{Entropy and temperature.} The module
        \texttt{lib/entropy.py} implements the definition
        $\SM = k \Min H(q)$ in \eqref{eq:SM_def} with
        $H(q) = -\sum_i \tilde q_i \log \tilde q_i$ where
        $\tilde q_i = \max\{q_i,0\} / \sum_j \max\{q_j,0\}$, and also
        computes
        $S_{\mathrm{M,in}} = k M_{\mathrm{in}} H(q)$,
        $S_{\mathrm{M,out}} = k M_{\mathrm{out}} H(q)$,
        $S_{\mathrm{M,hat}} = H(q)/\log K$ (with $K$ buckets), and optional
        per-bucket contributions
        $S_{\mathrm{M,in}}^{(i)} = k M_{\mathrm{in}} \tilde q_i$.
        The module \texttt{lib/temperature.py} builds $\TL$ from credit
        spreads, depth, and turnover by forming
        $z_s = Z(1/(\mathrm{spread}+\epsilon))$,
        $z_d = Z(1/(\mathrm{depth}+\epsilon))$,
        $z_t = Z(\mathrm{turnover})$, combining
        $\hat T = z_s z_d (1 + \tfrac{1}{2} z_t)$, and finally
        min--max normalising to obtain
        $T_{\mathrm{L}} = (\hat T - \min \hat T) / (\max \hat T - \min \hat T)$.
  \item \textbf{Internal-energy gauge.}
        Within the credit tables the column $U(t)$ is used as the
        internal-energy / capacity proxy entering \eqref{eq:first_law},
        \eqref{eq:FC_def}, and \eqref{eq:XC_def}.
        When a dedicated $U$ series is not available, the code fills it
        by combining available quantities in a conservative order
        (preferring an explicit $U$ column, then GDP-like series
        such as $U_{\text{gdp\_only}}$ or $Y$, and only as a last resort
        falling back to $L_{\mathrm{real}}$).
        An optional detrended version $U^\star$ is produced by subtracting
        a no-lookahead rolling mean so that diagnostics can focus on
        fluctuations around a slowly varying baseline.
  \item \textbf{Depth and turnover enrichment.}
        When region-specific depth/turnover series are not supplied, the
        helper \texttt{lib/credit\_enrichment.py} constructs depth and
        turnover heuristically from real credit stocks $L_{\mathrm{real}}$
        and capacity proxy $U$ as described in the main text.
  \item \textbf{Capacity, headroom, and coupling.} The core constructor
        \texttt{lib/indicators.py} merges money, allocation, credit, and
        regulatory tables, renames regulatory quantities into
        $\VC$ and $\pC$, and, when configured with the ``min\_headroom''
        formula, sets
        $\VC(t) = \min_j H_j(t)$ over headroom components.
        External pressure/temperature drivers are aggregated in\\
        \texttt{lib/external\_coupling.py}
        into $E_p$ and $E_T$; when the
        coupling coefficients $\alpha,\delta$ are non-zero the code applies
        the affine shifts
        $p_{\mathrm{C}} \leftarrow p_{\mathrm{C}}^{\text{baseline}} +
        \alpha E_p$ and
        $T_{\mathrm{L}} \leftarrow T_{\mathrm{L}}^{\text{baseline}} +
        \delta E_T$.
        The same module computes $\FC$, $\XC$, and the Gibbs-style $\GC$
        and streams the loop-area estimator in $(\pC,\VC)$.
  \item \textbf{Diagnostics.}
        The diagnostics routine in \texttt{lib/indicators.py} 
        estimates $\partial S_{\mathrm{M}}/\partial V_{\mathrm{C}}|_{T_{\mathrm{L}}}$
        and $\partial p_{\mathrm{C}}/\partial T_{\mathrm{L}}|_{V_{\mathrm{C}}}$
        via rolling least squares, and constructs finite-difference versions
        of the first-law decomposition by computing
        $\Delta U$, $\Delta S_{\mathrm{M}}$, $\Delta V_{\mathrm{C}}$,
        $\bar T_{\mathrm{L}}$, $\bar p_{\mathrm{C}}$, and the implied
        $Q_{\text{like}}$ and $W_{\text{like}}$ series.
  \item \textbf{Per-region orchestration.}
        Scripts \texttt{scripts/02\_compute\_indicators.py},\\
        \texttt{scripts/02\_compute\_indicators\_eu.py},\\
        \texttt{scripts/02\_compute\_indicators\_us.py}, and the helper
        \texttt{lib/regions.py} assemble the inputs for JP/EU/US and write
        \texttt{site/indicators*.csv} used by the dashboard and tests.
  \item \textbf{Fallbacks and CI reproducibility.} To keep tests and
        scheduled builds deterministic, \texttt{scripts/ci\_prepare\_minimal\_data.py}
        seeds tiny synthetic CSVs when real inputs are absent, and the
        continuous-integration workflows (GitHub Actions) re-run the
        pipeline under pinned dependencies so that indicators and the
        report can be regenerated from a fixed environment.
\end{itemize}

\section*{Appendix: Glossary (For Physicists New to Finance)}

\paragraph{Numbering.}
Each term is assigned a unique K-number (K1, K2, \ldots) in the order it is
first referenced in the main text. Use the superscript K\# markers to jump
back here; headings show the same inline K\#.

This glossary explains the main banking, credit, and regulatory terms used
in this note for readers with a physics background. Definitions are informal
and intuition-friendly. When helpful, we include analogies to thermodynamics
or dynamical systems. These are aids to intuition, not strict identities.

\subsection*{1.\ Regulation and Bank Balance Sheets}

\paragraph{K1 CET1 (Common Equity Tier 1).}
Core equity capital of a bank (common shares, retained earnings). It absorbs
losses and underpins solvency. Analogy: the bank's fundamental ``energy
reserve''.

\paragraph{K2 RWA (Risk-Weighted Assets).}
Total assets weighted by regulatory risk factors. Riskier exposures receive
higher weights. The ratio CET1/RWA is a key prudential metric. Analogy: an
effective load or mass adjusted for fragility.

\paragraph{K3 CET1/RWA ratio and slack.}
CET1/RWA indicates how much high-quality capital backs risk-weighted assets.
Regulation sets a minimum. The slack is the excess above this minimum and
represents capacity to expand or absorb shocks. Analogy: safety margin or
unused headroom in a constrained system.

\paragraph{K4 Credit capacity / Headroom ($\VC$).}
Effective room for additional lending, given capital, liquidity, and internal
limits. In this note $\VC$ is the state-like variable capturing this
remaining capacity. Analogy: effective volume available before hitting hard
walls.

\paragraph{K5 Credit pressure ($\pC$).}
Shadow price of capacity: the marginal cost or value of relaxing or
tightening $\VC$. Defined in analogy to
$\pC = -(\partial U / \partial \VC)_{\SM}$ or $-\partial F / \partial \VC$.
Analogy: thermodynamic pressure; tighter constraints $\Rightarrow$ higher
$\pC$.

\paragraph{K6 HQLA (High Quality Liquid Assets).}
Very liquid, low-risk assets (e.g.\ government bonds) that can be sold
quickly in stress. They support liquidity ratios and crisis resilience.
Analogy: high-grade stored energy that can be tapped on demand.

\paragraph{K7 Liquidity buffer.}
Cash plus HQLA held as an emergency reserve to withstand sudden outflows or
market stress. Analogy: a backup battery or safety reservoir.

\paragraph{K8 LCR (Liquidity Coverage Ratio).}
Regulatory ratio: HQLA divided by projected net cash outflows over a
30-day stress scenario. It tests short-term liquidity robustness. Analogy:
can the system run for 30 days under worst-case load?

\paragraph{K9 NSFR (Net Stable Funding Ratio).}
Regulatory ratio over a one-year horizon: stable funding relative to
required stable funding. It limits excessive maturity mismatch. Analogy:
ensuring long-lived assets are backed by sufficiently slow-decaying sources.

\subsection*{2.\ Money, Credit, and Flow Concepts}

\paragraph{K10 Money-in-circulation.}
An operational measure of money that is actually circulating in the selected
system and horizon (e.g.\ deposits used in payments), not merely base money
on the central bank balance sheet. Analogy: active particles taking part in
interactions.

\paragraph{K11 Credit stocks and flows.}
Stocks: outstanding amounts of loans or credit at a given time. Flows:
changes per period (new lending, repayments, net issuance). Analogy:
internal energy vs.\ power / energy flux.

\paragraph{K12 Margin credit / Securities financing.}
Credit used to finance positions in securities (e.g.\ margin loans, certain
repos). It amplifies leverage in financial markets and is often separated
from credit to the real economy.

\paragraph{K13 Commercial Paper (CP).}
Short-term unsecured debt issued by firms. Provides funding outside
traditional bank loans. Analogy: an alternative branch in the credit
circuit.

\paragraph{K14 Policy work ($W_{\mathrm{policy}}$).}
Deliberate interventions by central banks, regulators, or governments:
interest rate changes, asset purchases, guarantees, capital rule
adjustments, and similar actions. In our decomposition this is the
structured, intentional part of ``work''. Analogy: externally applied work
on the system.

\subsection*{3.\ Structural and Analytical Concepts}

\paragraph{K15 MECE (Mutually Exclusive, Collectively Exhaustive).}
A rule for defining categories: (i) no overlaps between categories,
(ii) no gaps in coverage. For the entropy
$\SM = k \Min H(q)$, the allocation shares $q_i$ should be MECE, otherwise
$\SM$ is distorted by double counting or omissions.

\paragraph{K16 State variable and proxy.}
A state variable is determined by the current state, not by the detailed
path. In practice, finance uses observable proxies (e.g.\ ratios, indices)
to approximate such variables. Our Maxwell-like relation
\eqref{eq:Maxwell} tests whether chosen proxies behave as if they came from
a consistent potential $U(\SM,\VC,\ldots)$.

\paragraph{K17 Hysteresis and loop area.}
If trajectories in the $(\pC,\VC)$ plane form loops with non-zero area over
policy or credit cycles, this indicates irreversibility or
dissipation-like effects (e.g.\ stress, misallocation, or path dependence).
Analogy: magnetic hysteresis, frictional cycles, or inelastic processes.

\paragraph{K18 Stress testing and early-warning indicators.}
Stress tests simulate severe but plausible scenarios to see whether banks or
systems survive. Early-warning indicators attempt to flag fragility in
advance. Within this framework, quantities like $\FC$, $\XC$, and loop
areas are intended as structured candidates to complement such tools.

\paragraph{K19 Money vs.\ credit (QTM vs.\ QTC perspective).}
In traditional Quantity Theory of Money (QTM), money is primary and credit
is often treated as derived. In the Quantity Theory of Credit (QTC) and in
this note, credit creation is primary: bank lending creates deposits, and
money is an accounting outcome of credit decisions. This shift motivates
tracking credit capacity and dispersion as state-like objects.

\begin{thebibliography}{13}

\bibitem{Fisher1911}
Irving Fisher.
\newblock \emph{The Purchasing Power of Money}.
\newblock Macmillan, 1911.

\bibitem{Friedman1956}
Milton Friedman.
\newblock The quantity theory of money -- a restatement.
\newblock In Milton Friedman, editor, \emph{Studies in the Quantity Theory of
  Money}. University of Chicago Press, 1956.

\bibitem{Laidler1985}
David Laidler.
\newblock \emph{The Demand for Money: Theories, Evidence, and Problems}.
\newblock Harper \& Row, 3rd edition, 1985.

\bibitem{Werner2012}
Richard~A. Werner.
\newblock Quantity theory of credit and empirical evidence on bank money
  creation.
\newblock \emph{International Review of Financial Analysis}, 2012.

\bibitem{BoE2014}
Michael McLeay, Amar Radia, and Ryland Thomas.
\newblock Money creation in the modern economy.
\newblock \emph{Bank of England Quarterly Bulletin}, 2014.

\bibitem{Theil1967}
Henri Theil.
\newblock \emph{Economics and Information Theory}.
\newblock North-Holland, 1967.

\bibitem{Foley1994}
Duncan~K. Foley.
\newblock A statistical equilibrium theory of markets.
\newblock \emph{Journal of Economic Theory}, 1994.

\bibitem{DragulescuYakovenko2000}
Adrian Dr\u{a}gulescu and Victor~M. Yakovenko.
\newblock Statistical mechanics of money.
\newblock \emph{European Physical Journal B}, 2000.

\bibitem{Shannon1948}
Claude~E. Shannon.
\newblock A mathematical theory of communication.
\newblock \emph{Bell System Technical Journal}, 1948.

\bibitem{Jaynes1957}
Edwin~T. Jaynes.
\newblock Information theory and statistical mechanics.
\newblock \emph{Physical Review}, 1957.

\bibitem{Wall1977}
G{\"o}ran Wall.
\newblock Exergy -- a useful concept.
\newblock Technical report, Chalmers University, 1977.

\bibitem{Li2024}
Wei Li, Jingyi Yang, and Jian Chen.
\newblock Market temperature and entropy: A microstructure-based view of liquidity.
\newblock \emph{Entropy}, 2024.

\bibitem{Feng2019}
Jean Feng.
\newblock A thermodynamic picture of financial market and model risk.
\newblock \emph{arXiv preprint arXiv:1909.XXXXX}, 2019.

\end{thebibliography}

\end{document}
